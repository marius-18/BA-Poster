\fett{\cb{}} ist  ein Prozess, bei dem Kanten zufällig getauscht werden. Bei einem \ct{} werden
zwei verschiedene Knoten $u$ und $v$ ($u\neq v$) zufällig uniform verteilt ausgewählt und deren disjunkte Nachbarschaft
zufällig durchmischt. 
Dabei muss darauf geachtet werden, dass durch das Tauschen keine Eigenschleifen oder Multikanten
entstehen und dass die Bipartitheit nicht verletzt wird.
Ein Beispiel ist in Abbildung \ref{fig:curveball_trade_graph} gegeben.
%
%
%
%%%%% CURVEBALL TRADE auf graph
\begin{figure}[H]
\centering
\begin{tikzpicture}
		\tikzset{node style/.style={shape=circle,draw=black, inner sep=5pt,}}
                                
\node[node style] at (0, -0.5)     (1)     {$v_1$};
\node[node style] at (0, -2.5)   (2)     {$v_2$};


\node[node style, fill=sandgrau] at (3, 0.5)     (5)     {$v_3$};
\node[node style, fill=sonnengelb] at (3, -1.5)   (6)     {$v_4$};
\node[node style, fill=sonnengelb] at (3, -3.5)   (7)     {$v_5$};


   
\draw[line width=0.1mm, >=latex]
            (1)     edge[right]    node {} (5)
            (1)     edge[right]    node {} (7)
            (2)     edge[right]    node {} (5)
            (2)     edge[right]    node {} (6)

;
\end{tikzpicture}
\begin{tikzpicture}
\node[] at (0, -1.3)   (1)     {};
\node[] at (2, -1.3)   (5)     {};
\node[] at (0, -3.65)   (7)     {};

	\draw[line width=0.8mm, >=latex, <->]
            (1)     edge[right]    node {} (5);

\end{tikzpicture}
\begin{tikzpicture}
\tikzset{node style/.style={shape=circle,draw=black, inner sep=5pt,}}
                                
\node[node style] at (0, -0.5)     (1)     {$v_1$};
\node[node style] at (0, -2.5)   (2)     {$v_2$};

\node[node style, fill=sandgrau] at (3, 0.5)     (5)     {$v_3$};
\node[node style, fill=sonnengelb] at (3, -1.5)   (6)     {$v_4$};
\node[node style, fill=sonnengelb] at (3, -3.5)   (7)     {$v_5$};


\draw[line width=0.1mm, >=latex]
            (1)     edge[right]    node {} (5)
            (1)     edge[right]    node {} (6)
            (2)     edge[right]    node {} (5)
            (2)     edge[right]    node {} (7)

;
\end{tikzpicture}
\caption{Auf einem der beiden Graphen wird ein \ct{} auf den Knoten $v_{1}$ und $v_{2}$ ausgeführt. 
In der grau markierten gemeinsamen Nachbarschaft liegt $v_{3}$, 
die disjunkte Nachbarschaft $\{v_{4},v_{5}\}$ ist in gelber Farbe gekennzeichnet.
In diesem Beispiel gibt es nur die zwei dargestellten Graphen, die durch Tauschen der disjunkten 
Nachbarschaft entstehen können. Ein \ct{} würde jeweils mit Wahrscheinlichkeit 0.5 einen der beiden 
Graphen zurückgeben.}
\label{fig:curveball_trade_graph}
\end{figure}
%
%
%

Ein \fett{\gc{} Tausch} besteht aus mehreren \ct{en}, wobei möglichst 
jeder Knoten Teil eines solchen \ct{es} sein soll. 





Im Fall von bipartiten Graphen werden die \ct{e} lediglich auf den Knoten 
aus einer der beiden Partitionsklassen ausgeführt. Diese wird als \fett{aktive} Partition
bezeichnet. 


Somit wird verhindert, dass durch einen \gc{} Tausch Eigenschleifen entstehen oder
die Bipartitheit verletzt wird. Auch werden dadurch unnötige \ct{e} verhindert.









%
%
% 
%%%%%% Global Curveball auf partitionsarray
\begin{figure}[H]
\centering
  \begin{tikzpicture}[decoration=brace]
      
      
    %% COMMON FÄRBEN  
    \foreach \x in {0,2,4,6,8,10,12}
		{
			\fill [ fill =sandgrau, draw =black ]  (\x ,0) rectangle (\x+2 ,2) ;
		};
    
	%\node[] at (-1.8, 0.5)     (5)     {\partvek};
    
    % untere geschweifte Klammer mit Text darunter:
    \foreach \x in {0,4,8} 
 		\draw[bend angle=60,bend right,  <->,>=latex, very thick] (\x+1,0) to  node[below= 1ex] {\cb{}} (\x+3,0) ;
	
	
	\draw[bend angle=60,bend right,  <->,>=latex, very thick, draw=none] (14,0) to  node[below= 1ex] {\textcolor{black}{kein Tausch}} (16,0) ;

	\draw[<-,>=latex, very thick] (13,0) to  node[below= 1ex] {} (15.0,-1) ;
%\node[] at (11.0, -0.7)     (5)     {kein Tausch};

  \end{tikzpicture}
  \caption{\gc{} auf dem zufällig permutierten Array der aktiven Partition. Da die Anzahl der Elemente in
  der aktiven Permutation ungerade ist, wird auf einem Knoten kein \ct{} ausgeführt.}
  \label{fig:global_curveball_trade_vector}
  
\end{figure}
%
%
%

