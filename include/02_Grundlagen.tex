\fett{\cb{}} ist  ein Prozess, bei dem Kanten zufällig getauscht werden. Bei einem \ct{} werden
zwei verschiedene Knoten $u$ und $v$ zufällig uniform verteilt ausgewählt und deren disjunkte Nachbarschaft
zufällig durchmischt. 
Dabei muss im Allgemeinen darauf geachtet werden, dass durch das Tauschen keine Eigenschleifen oder Multikanten
entstehen und dass die Bipartitheit nicht verletzt wird.
Ein Beispiel ist in Abbildung \ref{fig:curveball_trade_graph} gegeben.
%
%
%
%%%%% CURVEBALL TRADE auf graph
\begin{figure}[H]
\centering
\begin{tikzpicture}
		\tikzset{node style/.style={shape=circle,draw=black, inner sep=5pt,}}
                                
\node[node style,double = dunkelgrau] at (0, -0.5)     (1)     {\small$v_1$};
\node[node style,double = dunkelgrau] at (0, -2.5)   (2)     {\small$v_2$};


\node[node style, fill=sandgrau] at (3, 0.5)     (5)     {\small$v_3$};
\node[node style, fill=sonnengelb] at (3, -1.5)   (6)     {\small$v_4$};
\node[node style, fill=sonnengelb] at (3, -3.5)   (7)     {\small$v_5$};


   
\draw[line width=0.1mm, >=latex]
            (1)     edge[right]    node {} (5)
            (1)     edge[right]    node {} (7)
            (2)     edge[right]    node {} (5)
            (2)     edge[right]    node {} (6)

;
\end{tikzpicture}
\begin{tikzpicture}
\node[] at (0, -1.3)   (1)     {};
\node[] at (2, -1.3)   (5)     {};
\node[] at (0, -3.65)   (7)     {};

	\draw[line width=0.8mm, >=latex, <->]
            (1)     edge[right]    node {} (5);

\end{tikzpicture}
\begin{tikzpicture}
\tikzset{node style/.style={shape=circle,draw=black, inner sep=5pt,}}
                                
\node[node style,double = dunkelgrau] at (0, -0.5)     (1)     {\small$v_1$};
\node[node style,double = dunkelgrau] at (0, -2.5)   (2)     {\small$v_2$};

\node[node style, fill=sandgrau] at (3, 0.5)     (5)     {\small$v_3$};
\node[node style, fill=sonnengelb] at (3, -1.5)   (6)     {\small$v_4$};
\node[node style, fill=sonnengelb] at (3, -3.5)   (7)     {\small$v_5$};


\draw[line width=0.1mm, >=latex]
            (1)     edge[right]    node {} (5)
            (1)     edge[right]    node {} (6)
            (2)     edge[right]    node {} (5)
            (2)     edge[right]    node {} (7)

;
\end{tikzpicture}
\caption{Auf einem der beiden Graphen wird ein \ct{} auf den Knoten $v_{1}$ und $v_{2}$ ausgeführt. 
Die gemeinsame Nachbarschaft ist in grau gekennzeichnet, 
die disjunkte Nachbarschaft in gelber Farbe.
In diesem Beispiel gibt es nur die zwei dargestellten Graphen, die durch Tauschen der disjunkten 
Nachbarschaft entstehen können.}
\label{fig:curveball_trade_graph}
\end{figure}
%
%
%

Ein \fett{\gc{} Tausch} besteht aus mehreren \ct{en}, wobei möglichst 
jeder Knoten Teil eines solchen \ct{es} sein soll. 
Im Fall von bipartiten Graphen werden die \ct{e} lediglich auf den Knoten 
aus einer der beiden Partitionsklassen (diese wird \fett{aktiv} genannt) ausgeführt.
Somit wird verhindert, dass durch einen \gc{} Tausch Eigenschleifen entstehen oder
die Bipartitheit verletzt wird. Auch werden dadurch unnötige \ct{e} vermieden, welche den Graph nicht verändern würden.
Auf Grund der Bipartitheit überschneiden sich 
die einzelnen \ct{e} %überschneiden sich \red{im bipartiten Fall nicht} 
nicht und können daher vollständig parallel ausgeführt werden.

%
%
% 
%%%%%% Global Curveball auf partitionsarray
\begin{figure}[H]
\centering
  \begin{tikzpicture}[decoration=brace]
      
      
    %% COMMON FÄRBEN  
    \foreach \x in {0,2,4,6,8,10,12}
		{
			\fill [ fill =sandgrau, draw =black ]  (\x ,0) rectangle (\x+2 ,2) ;
		};
    
	%\node[] at (-1.8, 0.5)     (5)     {\partvek};
    
    % untere geschweifte Klammer mit Text darunter:
    \foreach \x in {0,4,8} 
 		\draw[bend angle=60,bend right,  <->,>=latex, very thick] (\x+1,0) to  node[below= 1ex] {\small \cb{}} (\x+3,0) ;
	
	
	\draw[bend angle=60,bend right,  <->,>=latex, very thick, draw=none] (14,0) to  node[below= 1.6ex] {\small \textcolor{black}{kein Tausch}} (15,0) ;

	\draw[<-,>=latex, very thick] (13,0) to  node[below= 1ex] {} (14.0,-1) ;
%\node[] at (11.0, -0.7)     (5)     {kein Tausch};

  \end{tikzpicture}
  \caption{\gc{} auf dem zufällig permutierten Array der aktiven Partition. Bei ungerader Anzahl wird ein Knoten ausgelassen.}
  \label{fig:global_curveball_trade_vector}
  
\end{figure}
%
%
%

