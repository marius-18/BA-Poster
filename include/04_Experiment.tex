In Abbildung \ref{fig:messung_slowdown} lassen sich die Laufzeiten der verschiedenen Varianten vergleichen.
Dabei wird für jede Instanz \red{das Verhältnis der Laufzeit der jeweiligen Variante zur geringsten Laufzeit berechnet.}
Die einzelnen Slowdowns sind für jede Variante aufsteigend sortiert und als Kurve eingezeichnet.
Bei einer guten Methode verläuft die Kurve möglichst lange nahe der 1 und steigt nicht stark an.
Die beiden besten Methoden sind \red{markiert}.


\begin{figure}[H]
\centering
	\includegraphics[width = 0.49\textwidth]{../BA-LaTeX/figures/slowdown.pdf}
	\caption[Slowdown der einzelnen Varianten im jeweiligen Vergleich zur Variante mit der geringsten Laufzeit.]
	{Slowdown der einzelnen Varianten im jeweiligen Vergleich zur Variante mit der geringsten Laufzeit.}
	\label{fig:messung_slowdown}
\end{figure}

Die Abbildung \ref{fig:messung_small} zeigt einen direkten Vergleich der zwei besten Methoden auf ausgewählten Instanzen.
Während die Laufzeiten bei den meisten Instanzen nahezu identisch sind, existieren einige Instanzen, bei denen die
in blau gekennzeichnete Variante eine deutlich geringere Laufzeit aufweist. Deswegen wird diese Variante %zur Implementierung eines \ct{es}
ausgewählt.

\begin{figure}[H]
\centering
	\includegraphics[width = 0.47\textwidth]{../BA-LaTeX/figures/small_aufsteigend.pdf}
	\caption[Laufzeitvergleich der zwei besten Varianten auf ausgewählten Instanzen] 
	{Vergleich der Laufzeiten der zwei besten Varianten. Auf der horizontalen Achse sind die Instanzen mit der jeweiligen Größe der beiden Arrays aufgetragen.}
	\label{fig:messung_small}
\end{figure}
