Für einen \ct{} muss also die gemeinsame und disjunkte Nachbarschaft bestimmt 
und die disjunkte Nachbarschaft zufällig durchmischt werden.
Die Nachbarschaften eines jeden Knotens sind jeweils in einem Array gespeichert. 
Dies ist in Abbildung \ref{fig:curveball_trade_vector} skizziert.
%
%
%
%
%
%
%%%%% CURVEBALL TRADE auf Array
\begin{figure}[H]
\centering
  \begin{tikzpicture}[decoration=brace]
      
      
    %% COMMON FÄRBEN  
    \foreach \x in {0,1,2,3}
		{
			\fill [ fill =lightgray, draw =black ]  (\x ,0) rectangle (\x+1 ,1) ;
			\fill [ fill =lightgray, draw =black ]  (\x ,-2.5) rectangle (\x+1 ,-1.5) ;
		};

    %% DISJOINT OBEN FÄRBEN  
    \foreach \x in {4,5,6,7,8,9}
		{
			\fill [ fill =sonnengelb, draw =black ]  (\x ,0) rectangle (\x+1 ,1) ;
		};
		
	%% DISJOINT UNTEN FÄRBEN  
    \foreach \x in {4,5,6,7,8,9,10,11,12}
		{
			\fill [ fill =sonnengelb, draw =black ]  (\x ,-2.5) rectangle (\x+1 ,-1.5) ;
		};
    
\node[] at (-1.0, 0.5)     (5)     {\small $N(u)$};
\node[] at (-1.0, -2)     (5)     {\small $N(v)$};
    
    \draw[decorate, yshift=+2ex,decoration={brace,amplitude=5pt}] (9.8,-0.9) -- node[below=0.4ex] {} (4.2,-0.9);

    \draw[decorate, yshift=+1ex, decoration={brace,amplitude=5pt}] (4.2,-1.8) -- node[below=0.4ex] {} (12.8,-1.8);

    
    % untere geschweifte Klammer mit Text darunter:
    \draw[decorate, yshift=-1ex, decoration={brace,amplitude=8pt}] (3.8,-2.3) -- node[below=0.7ex] {\small gemeinsame-} (0.2,-2.3);
    \draw[decorate, yshift=-1ex, decoration={brace,amplitude=8pt}] (12.8,-2.3) -- node[below=0.7ex] {\small disjunkte Nachbarschaft} (4.2,-2.3);


	\draw[out=-70, in=110, <->,>=latex, very thick] (7,-0.3) to  node[right= 3ex] {\small vertauschen} (8.5,-1.1) ;

  \end{tikzpicture}
  \caption{Skizze eines \ct{es} auf den Arrays}
  \label{fig:curveball_trade_vector}

\end{figure}
%
%
%
%
Zur Bestimmung der gemeinsamen und disjunkten Nachbarschaft werden insgesamt 7 verschiedene Methoden betrachtet, zum zufälligen Tauschen 2 weitere.
Weiterhin wird geprüft, ob es zu einem Laufzeitvorteil führt, die Arrays der Nachbarschaften immer sortiert zu halten. Auf diese Weise
entstehen insgesamt 28 verschiedene Methoden einen \ct{} durchzuführen. 
Durch eine experimentelle Untersuchung auf verschiedenen Instanzen wird bestimmt, welche Methode
das beste Laufzeitverhalten aufweist.
