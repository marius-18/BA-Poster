Bei der Analyse komplexer Netzwerke, wie beispielsweise soziale Netzwerke, 
werden die zugrundeliegenden Graphen häufig mit zufälligen Graphen verglichen, 
um deren Struktur zu untersuchen\cite{DBLP:conf/esa/CarstensH0PTW18}.
%
Zum Erzeugen von zufälligen Graphen existieren diverse Modelle wie 
beispielsweise der Erd\H{o}s-R\'enyi-Graph \cite{erdos}
oder der Gilbert-Graph \cite{gilbert}.
Diese Graphen
weisen jedoch in der Regel kaum eine Ähnlichkeit zu dem zu analysierenden Netzwerk auf.
Deshalb verwendet man Zufallsgraphen, 
bei welchen jeder Knoten denselben Grad wie im originalen Graphen hat.
\cb{} \cite{curveball} und \gc{}\cite{DBLP:conf/esa/CarstensH0PTW18} sind Lösungsansätze hierfür.

Ziel dieser Bachelorarbeit ist die Anpassung von \gc{} an bipartite Graphen zur 
Reduzierung der Laufzeit. Der entwickelte Algorithmus wird Teil des OpenSource Projekts \nk{}\cite{nk_page} werden.
