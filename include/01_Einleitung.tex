Ziel dieser Bachelorarbeit ist es, einen effizienten Algorithmus
zur Randomisierung massiver bipartiter Graphen zu entwickeln.
Die Graph Randomisierung ist vor allem zur Analyse von großen Netzwerken eine häufig verwendete Methode.
 Dazu 
wurde das Konzept des \gc{} auf bipartiten Graphen
angepasst. Es werden verschiedene Algorithmen zur Umsetzung diskutiert.
Anhand von Benchmarks wird unter den getesteten Methoden diejenige ausgewählt,
welche die geringste Laufzeit aufweist.
Im Vergleich zu dem schon existierenden \cb{} Algorithmus 
wird mit dem in dieser Arbeit entwickelten Algorithmus 
auf manchen Testinstanzen ein Speedup von bis zu 17 auf einem Prozessor mit 8 Kernen und Hyperthreading 
erreicht. Selbst ohne Parallelisierung wird mit einer sequenziellen Version des bipartiten \gc{} ein Speedup
von bis zu 2 gegenüber \cb{} erreicht.

