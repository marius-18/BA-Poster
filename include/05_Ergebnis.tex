Abschließend wird der neu entwickelte bipartite \gc{} mit dem schon existierenden \cb{} \red{auf verschiedenen Graphen} verglichen.
Die gemessenen Laufzeiten sind in Abbildung \ref{fig:speedup_komplett} dargestellt. Auf den Instanzen mit 
weit mehr als 10.000 Knoten zeigt der bipartite \gc{} eine deutlich bessere Laufzeit. Es wird ein Speedup von 
bis zu 17 erreicht. Dies liegt vor allem an der Parallelität im bipartiten \gc{}. Aber auch ohne
die Parallelisierung erreicht der sequenzielle bipartite \gc{} einen Speedup von ungefähr 2.



\begin{figure}[H]
\centering
	\includegraphics[width = 0.45\textwidth]{../BA-LaTeX/figures/speedup.pdf}
	\caption[Laufzeitvergleich von bipartitem \gc{} und einer abgeänderten Variante von \cb{}]
	{Vergleich von bipartitem \gc{} und \cb{}.
	Als Beschriftung dient die Anzahl an Knoten, Kanten und aktiven Knoten.}
	\label{fig:speedup_komplett}
\end{figure} 
   
